\documentclass[11pt]{article}
\usepackage{amsmath}
\usepackage{amsfonts} 
\usepackage{enumitem}
\usepackage{mathtools}

\title{Notes on Introductory Series}
\author{John Hunter Kohler}
\date{March 2020}


\newcommand{\define}{\coloneqq}


\begin{document}
\maketitle

\quad Though the formal definition of sequences and series limits will not be given. The notation used will be standard and 
informal. No proofs will be given similar to the entirety of this document. Assume all sequences and series are real.

\section*{Foundation of Sequences}
	This secition focuses on the basics of sequences and methods of computation on the infinite terms of some sequence. 
	A sequence $\{a_n\}$ is said to be convergent if
		\begin{equation}
			\lim_{n\to\infty} a_n \in\mathbb{R}
		\end{equation}
	If the above property is not held, the sequence is said to be divergent. When computing limits of two 
	convergent sequences $\{a_n\},\{b_n\}$, it is possible to seperatly evaluate the sequences.
		\begin{equation}
			\lim_{n\to\infty} f(a_n,b_n) = f(\lim_{n\to\infty} a_n, \lim_{n\to\infty} b_n)
		\end{equation}
	A simple tool for the evaluation of limits upon convergent series is the squeeze theorem for series.
		\begin{gather}
			a_n <= c_n <= b_n\,,\ \text{for} \ n > (N\in\mathbb{N}) \\ 
			\lim_{n\to\infty} a_n = \lim_{n\to\infty} b_n = L
			\implies \lim_{n\to\infty} c_n = L
		\end{gather}
	Sequences, like functions, are said to be monotonic if they are strictly increasing or strictly decreasing.
	Like functions, sequences may be bounded above or below by some constant. If a sequence is bounded and monotonic, 
	then it must be convergent\footnote{Obviously, the same can be said about those sequences which are simply strictly increasing and 
	bounded above as well as strictly decreasing and bounded below.}.

\section*{Convergence of Series}
	Here, let us take a look at simple rules for figuring if a series converges or diverges. Obviously a convergent series is the sum of
	every term of a series such that the value is finite. If some series does not exist or is infinite, it is said to be divergent.
	A simple proposition is that
		\begin{equation}
			\text{if} \ \sum a_n \ \text{converges, then} \ \lim_{n\to\infty} a_n = 0
		\end{equation}
	Now recognize two more properties of series. If some series $\sum a_n$ is convergent yet $\sum |a_n|$ is not, then the series $\sum a_n$
	is said to be conditionally convergent. If, along with $\sum |a_n|$, $\sum a_n$ is convergent, we say that the series is absolutly
	convergent. Otherwise, the series is divergent. From this it is important to recognize one thing: if a series is conditionally convergent,
	then when algabraically rearranging\footnote{For the purposes of computation.} its explicit terms will allow you to, 
	without logical error, compute the sum of the series as any real number.

	\subsection*{Geometric Sums and P-Series}
		Firstly define a partial sum -this usually uses confusing notation- as
			\begin{equation}
				S_n= \sum_{i = 0}^{n}a_i
			\end{equation}
		The lower bound of the sum may change based on context, but this is the general notation with the exception of the differing use of capitalization
		of $s$. For a geometric sum, based on a sequence where the ratio between each term is a constant, the partial sum formula is slightly unintuitive,
		though important. It is vital in the proof of the infinite sum, though that will not be exemplified.
			\begin{gather}
				\text{for $a,r$ constant} \ S_n= \sum_{i = 0}^{n} ar^i = \dfrac{a(1-r^n)}{1-r} \\
				\lim_{n\to\infty} S_n = \dfrac{a}{1-r} \ \text{when} \ |r| < 1 
			\end{gather}
		As you will see later, this is a form of a power series, with radius of convergence\footnote{Put simply, as expressed above, the infinite
		geometric sum only converges for $|r| < 1$.} one. The other series to look at of the form for which we know convergence is the p-series
		\footnote{The special case is $p = 1$ which is the harmonic series. Not discussed here.}. 
			\begin{equation}
				\sum_{n=1}^{\infty} \frac{1}{n^p} \ \text{converges for}\ p > 1
			\end{equation}

	\subsection*{Integral Test}
		If there is some continuous, positive, strictly decreasing function $f(x)$ on,
			\begin{align}
				&\ \ x\in[k,\infty)\ \text{and}\ f(n) = a_n,\, n\in\mathbb{N},\ \text{then}\\
				&\int_{k}^{\infty} f(x)dx\ \text{and}\ \sum_{k}^{\infty} a_n\ \text{converge or diverge together.}
			\end{align}

	\subsection*{Comparison Test}
		This test is similar to, and essentially a pre-requisite of, the last theorem: 
			\begin{align}
				\text{If there is some}\ &\sum a_n ,\, \sum b_n \ \text{for} \ a_n \geq b_n \geq 0,\,n\in\mathbb{N}, \\
				\text{then}\ &\sum_{n=1}^{\infty} a_n \ \text{and} \ \sum_{n=1}^{\infty} b_n \ \text{converge or diverge together.}
			\end{align}

	\subsection*{Limit Comparison Test}
		By taking the associated sequences of two series and compair there infinite terms, a similar association of converge as before can
		be formed\footnote{These associations will let us extend knowledge of convergence from what we have proved to unknowns more easily.
		Also this would give the option of working the more simple of the two series.}. 
		\begingroup
			\begin{align}
				\text{Let} \sum a_n,\,\sum b_n \ \text{such that}\ &a_n \geq 0,\, b_n > 0, \\
				n\in\mathbb{N},\ &\text{and let} \ C \define \lim_{n\to\infty} \dfrac{a_n}{b_n}\\
				\text{If} \ C \in\mathbb{R}^{+} \ \text{then} \ \sum a_n \ \text{and}\ &\sum b_n \\
				\text{conver}&\text{ge }\text{or diverge together.}
			\end{align}
		\endgroup

	\subsection*{Alternating Series Test}
		To deal with series of the form below, this test is created. Once used, if it is possible to do so, another test may be required
		to determine the properties of the subseries. An alternating series is of the form:
			\begin{equation}
				a_n=(-1)^{n}b_n \ \text{or} \ a_n=(-1)^{n+1}b_n \ \text{for} \ (b_n \geq 0 \ \text{or} \ b_n \leq 0),\,n\in\mathbb{N}
			\end{equation}
		Furthermore, the test states:
			\begin{gather}
				\text{If} \ \lim_{n\to\infty} b_n = 0 \ \text{and} \{b_n\} \ \text{is monotonic\footnotemark,} \ \sum_{n=1}^{\infty} a_n \ \text{is convergent.}
			\end{gather}
		\footnotetext{For ($b_n \geq 0$ and strictly decreasing), or ($b_n \leq 0$ and strictly increasing).}

	\subsection*{Ratio Test}
		The ratio test will be the best tool presented here to deal with sequences involving factorials as well as rational equations of polynomials.
		By reviewing the test below you will see that it takes a form in which, for the described cases, we may cancel out confounding terms.
			\begin{align}
				\text{Let} \ L \define &\lim_{n\to\infty} \left| \dfrac{a_{n+1}}{a_n} \right| \\
				\text{If} \ L < 1, &\sum a_n \ \text{is absolutly convergent.} \\
				\text{If} \ L > 1, &\sum a_m \ \text{is divergent.} \\
				\text{If} \ L = 1, &\text{ the test is inconclusive.}
			\end{align}

	\subsection*{Root Test}
		This test is similar to the ratio test. Almost\footnote{Most cases that do not abide by this principle should not be presented
		in an introductory course. It may be worthwile checking both either way as either one may have been applied incorrectly} 
		all sequences produce the same results in the ratio and root test. This test is best utilized for series similar to
		$a_n = (b_n)^n$. The test functions thus:
			\begin{align}
				\text{Let} \ L \define &\lim_{n\to\infty} \left| a_n \right|^{\frac{1}{n}} \\
				\text{If} \ L < 1, &\sum a_n \ \text{is absolutly convergent.} \\
				\text{If} \ L > 1, &\sum a_n \ \text{is divergent.} \\
				\text{If} \ L = 1, &\text{ the test is inconclusive.}
			\end{align}
\end{document}















